\documentclass[]{friggeri-cv}
\addbibresource{bibliography.bib}

\begin{document}

\header{Amiri }{Heni}{étudiant en 2\textsuperscript{éme} année stic}

%----------------------------------------------------------------------------------------
%	SIDEBAR SECTION
%----------------------------------------------------------------------------------------

\begin{aside} % In the aside, each new line forces a line break
\section{contact}
Cité Elghazela
Ariana 8122
Tunis
~
+216 (53) 748 801
~
\href{mailto:heni.amiri@gmail.com}{heni.amiri@gmail.com}
\href{http://tn.linkedin.com/in/heniamiri}{linkedin://heni.amiri}
\section{Languages}
anglais: execllant
francais: bilingue
arabe: maternelle
\section{Programmation}
Rust (Primary Lang)
Python, C/C++, Java
Shell Scripting (ZSH/BASH), SQL \& Assembleur
\section{Systèmes d'exploitations}
GNU/Linux,Windows
Unix(OpenBSD,FreeBSD)
\section{Réseaux \& Services}
GNS3, Packet Tracer
Netfilter IPtables
VPN, Apache,OpenSSL OpenSSH
TCP/IP, IProute2
Routage(OSPF, RIP)...
\section{Bureautique}
\LaTeX, LibreOffice
Microsoft Office
\end{aside}

%----------------------------------------------------------------------------------------
%	EDUCATION SECTION
%----------------------------------------------------------------------------------------

\section{formation}

\begin{entrylist}
%------------------------------------------------
\entry
{2011--Now}
{Licence {\normalfont appliqué en STIC}}
{ISET'COM, Tunis}
{Option: sécurité des réseaux informatiques}
%------------------------------------------------
\entry
{2010--2011}
{Baccalauréat {\normalfont en sciences expériemntales}}
{Lycée Khmir, Tunis}
{Mention bien}
%------------------------------------------------
\end{entrylist}

%----------------------------------------------------------------------------------------
%	WORK EXPERIENCE SECTION
%----------------------------------------------------------------------------------------

\section{expérience}

\begin{entrylist}
%------------------------------------------------
\entry
{2012--2013}
{Société Nationale d’Exploitation et de Distribution des Eaux (SONEDE)}
{Tunis}
{\emph{Stage Technicien} \\
Mise en place d’une infrastructure de supervision systéme "Open Source"\\
Outils utilisés:
\begin{itemize}
\item \href{www.nagios.org/}{Nagios}:~Logiciel libre de surveillance des réseaux et systèmes.
\item \href{https://www.snort.org/}{Snort}: Système de détection d'intrusion.
\item \href{www.ntop.org/}{Ntop}: Analyseur de traffic réseau.
\item \href {https://www.alienvault.com}{OSSIM}: Framework de "gestion de la sécurité de l’information"
\end{itemize}}
%------------------------------------------------
\entry
{2011--2012}
{Tunisie Télécom}
{Tunis}
{\emph{Stage Ouvrier} \\
Stage de découverte du milieu professionnel (Centre de Construction des Lignes)}
%------------------------------------------------
\end{entrylist}

%----------------------------------------------------------------------------------------
%	Project SECTION
%----------------------------------------------------------------------------------------

\section{projets}

\begin{entrylist}
%------------------------------------------------
\entry
{2015}
{Solution PAAS Open Source}
{Communauté OpenStack, Tunis}
{Création d’une solution PAAS (Plateform As A Service ) à base des conteneurs Linux (LXC/Docker) sur l’infrastructure de cloud computing libre OpenStack}
\entry
{2014}
{Data Over Sound}
{Projet Individuel}
{Transmission de données entre deux ordinateurs par ultrasons}
\entry
{2014}
{Mainteneur de paquet}
{Communauté en ligne Archlinux}
{Mainteneur de paquet du noyau Linux avec le patch \href{https://grsecurity.net/}{Grsecurity} dans la communauté Archlinux}
\entry
{2013}
{Traducteurs Temps Réel}
{Forum Social Mondial, Tunis}
{Installation des traducteurs temps réel à base des cartes Raspberry PI pour les conférences au Forum Social Mondial 2013.}
%------------------------------------------------
\end{entrylist}

%----------------------------------------------------------------------------------------
%	COMMUNICATION SKILLS SECTION
%----------------------------------------------------------------------------------------

%------------------------------------------------

%------------------------------------------------
%----------------------------------------------------------------------------------------
%	INTERESTS SECTION
%----------------------------------------------------------------------------------------

\section{intérêt}

\textbf{Professionnel:} Sécurité des applications Web, programmation systéme, ingenierie inverse, CTF (Capture The Flag), administration systéme, exploitation binaire, GNU/Linux, logiciels libres, virtualisation\ldots\\
\textbf{Personnel:} Cinema, films, livres \& romans, philosophie, mythologie\ldots

%----------------------------------------------------------------------------------------
%	PUBLICATIONS SECTION
%----------------------------------------------------------------------------------------

%\section{publications}

\printbibsection{article}{article in peer-reviewed journal} % Print all articles from the bibliography

\printbibsection{book}{books} % Print all books from the bibliography

\begin{refsection} % This is a custom heading for those references marked as "inproceedings" but not containing "keyword=france"
\nocite{*}
\printbibliography[sorting=chronological, type=inproceedings, title={international peer-reviewed conferences/proceedings}, notkeyword={france}, heading=subbibliography]
\end{refsection}

\begin{refsection} % This is a custom heading for those references marked as "inproceedings" and containing "keyword=france"
\nocite{*}
\printbibliography[sorting=chronological, type=inproceedings, title={local peer-reviewed conferences/proceedings}, keyword={france}, heading=subbibliography]
\end{refsection}

\printbibsection{misc}{other publications} % Print all miscellaneous entries from the bibliography

\printbibsection{report}{research reports} % Print all research reports from the bibliography

%----------------------------------------------------------------------------------------

\end{document}